% Define document class
\documentclass[twocolumn]{aastex631}
\usepackage{showyourwork}

% Begin!
\begin{document}

% Title
\title{Differentiable Waveforms for Gravitational Wave Data Analysis}

\newcommand{\OKC}{\affiliation{The Oskar Klein Centre, Department of Physics, Stockholm University, AlbaNova, SE-106 91 Stockholm, Sweden}} 
\newcommand{\NORDITA}{\affiliation{Nordic Institute for Theoretical Physics (NORDITA), 106 91 Stockholm, Sweden}}
\newcommand{\UdeM}{\affiliation{Département de Physique, Université de Montréal, 1375 Avenue Thérèse-Lavoie-Roux, Montréal, QC H2V 0B3, Canada}} 
\newcommand{\Mila}{\affiliation{Mila -- Quebec AI Institute, 6666 St-Urbain, \#200, Montreal, QC, H2S 3H1}} 
\newcommand{\CGP}{\affiliation{Center for Gravitational Physics, University of Texas at Austin, Austin, TX 78712, USA}}
\newcommand{\CCA}{\affiliation{Center for Computational Astrophysics, Flatiron Institute, New York, NY 10010, USA}}

% Author list
\author{Adam Coogan} \UdeM \Mila
\author{Thomas D.~P.~Edwards} \OKC \NORDITA
\author{Daniel Foreman-Mackey} \CCA
\author{Maximiliano Isi} \CCA
\author{Kaze W.~K.~Wong} \CCA
\author{Aaron Zimmerman} \CGP

% Abstract with filler text
\begin{abstract}
    Here we will discuss our implementation of differentiable waveforms and demonstrate their benefits for a variety of data analysis tasks.
\end{abstract}

\section{Introduction}
\label{sec:intro}

The discovery of gravitational waves (GWs) from inspiraling and mergering compact objects (COs) has revolutionized our understanding of both fundamental physics and astronomy.
Although the data volumes coming from GW detectors are relatively small, analyzing the data is a computationally demanding task.
In addition, this computational cost will substatially increase when next generation detectors come online.

Derivatives are ubiqitously useful throughout data analysis tasks.
However, in the field of GW data analysis, analytic derivatives of the necessary quantities (such as the likelihood) have historically been difficult to obtain.
Numerical derivatives also suffer from accuracy issues stemming from rounding or truncation errors.
Recent progress in automatic differentiation (AD) has.... \citep{Coogan:2022qxs}

In this paper we argue that differentiable waveforms will be a vital component for the future of GW data analysis.
In addition, we present \texttt{ripple}, a small GW \texttt{python} package containing differentiable implementations of some of the main waveforms currently used in LIGO and Virgo analyses.

\section{Differentiable Waveforms}
\label{sec:waveforms}

A variety of waveform families have been developed to accurately model the GW emission from COs. 
When the COs are relatively well separated, the dynamics of the system can be well approximated with post-Newtonian corrections to the... .
However, close to merger, numerical relatively simulations are required to accurately model the GW emission.
Unfortunately, these numerical simulations are computationally expensive and cannot be run in conjunction with data analysis.


\subsection{Benchmarks}
\label{subsec:benchmarks}

\section{Applications}
\label{sec:applications}

\subsection{Effectualness Calculations}
\label{sec:effectualness}

\subsection{Fisher Analysis}
\label{subsec:fisher}

\subsection{Derivative Based Samplers - Hamiltonian Monte Carlo}
\label{subsec:hmc}

\subsection{Fitting Waveform Coefficients}
\label{subsec:hmc}

\section{Acknowledgments}
This work was supported by collaborative visits funded by the Cosmology and Astroparticle Student and Postdoc Exchange Network (CASPEN).

\bibliography{bib}

\section{Appendix}
\label{sec:appendix}

\end{document}
