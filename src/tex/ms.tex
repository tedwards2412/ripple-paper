% Define document class
\documentclass[twocolumn]{aastex631}
\usepackage{showyourwork}

% Begin!
\begin{document}

% Title
\title{\texttt{ripple}: Differentiable Waveforms for Gravitational Wave Data Analysis}

\newcommand{\OKC}{\affiliation{The Oskar Klein Centre, Department of Physics, Stockholm University, AlbaNova, SE-106 91 Stockholm, Sweden}} 
\newcommand{\NORDITA}{\affiliation{Nordic Institute for Theoretical Physics (NORDITA), 106 91 Stockholm, Sweden}}
\newcommand{\UdeM}{\affiliation{Département de Physique, Université de Montréal, 1375 Avenue Thérèse-Lavoie-Roux, Montréal, QC H2V 0B3, Canada}} 
\newcommand{\Mila}{\affiliation{Mila -- Quebec AI Institute, 6666 St-Urbain, \#200, Montreal, QC, H2S 3H1}} 
\newcommand{\CGP}{\affiliation{Center for Gravitational Physics, University of Texas at Austin, Austin, TX 78712, USA}}
\newcommand{\CCA}{\affiliation{Center for Computational Astrophysics, Flatiron Institute, New York, NY 10010, USA}}

% Author list
\author{Adam Coogan} \UdeM \Mila
\author{Thomas D.~P.~Edwards} \OKC \NORDITA
\author{Daniel Foreman-Mackey} \CCA
\author{Maximiliano Isi} \CCA
\author{Kaze W.~K.~Wong} \CCA
\author{Aaron Zimmerman} \CGP

% Abstract with filler text
\begin{abstract}
    Here we will discuss our implementation of differentiable waveforms and demonstrate their benefits for a variety of data analysis tasks.
\end{abstract}

\section{Introduction}
\label{sec:intro}

% Intro outline
% - What are gravitational waves and what are data analysis tasks?
% - What are derivatives, and how can they be used?
% - What is automatic differentiation, and why has it become particularly useful now? JAX?
% - Discussion of what waveforms are most ameanable to being differentiable
% - In this paper

The discovery of gravitational waves (GWs) from inspiraling and mergering compact objects (COs) has revolutionized our understanding of both fundamental physics and astronomy.
Although the data volumes from GW detectors are relatively small, analyzing the data is a computationally demanding task.
In addition, this computational cost will substatially increase when next generation detectors come online.
The complexity begins before data gathering where one is required to generate banks of template waveforms which will be used for a matched-filter search~\citep{Owen:1998dk, Owen:1995tm}. 
Once potential candidates are found, parameter estimation (PE) is performed to extract the detailed source properties of each event.
In the most minimial example, this requires one to perform MCMC over a 15 dimensional parameter space for each event.
Beyond this simple scenario, more complex waveform models with many additional parameters may be used to test for deviations from General Relativity.
Finally, using the results of the PE, population synthesis models are used to constrain the progenitors systems from which the black holes we see merging today began their journey~\citep{Wong:2022flg}.
All of these tasks require significant compute.
In this paper, we will argue that differentiable waveforms (and more generally differentiable pipelines) can play a significant role in alleaviating this computational demand.

Derivatives are ubiqitously useful throughout data analysis tasks.
For instance, during PE, derivative information can be used to guide the sampling towards regions with higher likelihood values (e.g. in Hamiltonian Monte Carlo~\citep{2017arXiv170102434B} or Gradient Decsent).
This use of the gradients is particularly useful for high dimensional spaces.
Unfortunately, in the field of GW data analysis, analytic derivatives of the necessary quantities (such as the likelihood) have historically been difficult to obtain.
Numerical derivatives also suffer from accuracy issues stemming from rounding or truncation errors.
However, recent progress in automatic differentiation (AD) has shown promise in allowing general, fast derivative calcuations for gravitational waveforms~\citep{Coogan:2022qxs}.

Automatic Differentiaton is a family of methods used to compute the machine precision derivatives with little computational ovearhead. 
Its has become popular due to its requirement in NNs...
Core idea is chain rule.
Why JAX?

There are a variety of gravitational waveforms currently used in analysis pipelines.
They are generally structured into different families, the most common of which are: the effective one body, inspiral-merger-ringerdown phenom, and numerical GW....
Unfortunately, not all are well suited for easy implementation into 


In this paper we argue that differentiable waveforms will be a vital component for the future of GW data analysis.
In addition, we present \texttt{ripple}, a small GW \texttt{python} package containing differentiable implementations of some of the main waveforms currently used in LIGO and Virgo analyses. 
The remainder of this paper is structured as follows. 
In Sec.~\ref{sec:waveforms} we discuss the differentiable waveforms implemented in \texttt{ripple} and perform some benchmarks to demonstrate the speed and accruacy of the derivative calculations. 
In Sec.~\ref{sec:applications} we discuss four distinct applications using differentiable waveforms. 
First, a simple demonstration of accelerated effectualness calculations using gradient descent.
Second, we show that Fisher analyses are quickly enabled using.
Third, we implement differnetiable detector response functions and show an illutrative example Hamiltonian Monte Carlo
Finally, we illustrate how the fit coefficients that form part of the IMRPhenom waveform models could be improved by high dimensional fitting, enabled by a differentiable waveform. 
The differentiable waveforms can be found at \texttt{ripple}~\cite{ripple}.

\section{Differentiable Waveforms}
\label{sec:waveforms}

A variety of waveform families have been developed to accurately model the GW emission from COs. 
When the COs are relatively well separated, the dynamics of the system can be well approximated with post-Newtonian corrections to the... .
However, close to merger, numerical relatively simulations are required to accurately model the GW emission.
Unfortunately, these numerical simulations are computationally expensive and cannot be run in conjunction with data analysis.


\subsection{Benchmarks}
\label{subsec:benchmarks}

\section{Applications}
\label{sec:applications}

\subsection{Effectualness Calculations}
\label{sec:effectualness}

\subsection{Fisher Analysis}
\label{subsec:fisher}

\subsection{Derivative Based Samplers - Hamiltonian Monte Carlo}
\label{subsec:hmc}

\subsection{Fitting Waveform Coefficients}
\label{subsec:hmc}

\section{Acknowledgments}
This work was supported by collaborative visits funded by the Cosmology and Astroparticle Student and Postdoc Exchange Network (CASPEN).

\bibliography{bib}

\section{Appendix}
\label{sec:appendix}

\end{document}
