% Define document class
\documentclass[twocolumn]{aastex631}
\usepackage{showyourwork}

% Begin!
\begin{document}

% Title
\title{Differentiable Waveforms for Gravitational Wave Data Analysis}

% Author list
\author{Adam Coogan}
\author{Thomas D.~P.~Edwards}
\author{Daniel Foreman-Mackey}
\author{Maximiliano Isi}
\author{Kaze W.~K.~Wong}
\author{Aaron Zimmerman}

% Abstract with filler text
\begin{abstract}
    Here we will discuss our implementation of differentiable waveforms and demonstrate their benefits for a variety of data analysis tasks.
\end{abstract}

\section{Introduction}
\label{sec:intro}

Derivatives are ubiqitously useful throughout data analysis tasks.
However, in the field of gravitational wave (GW) data analysis, they have historically been difficult to obtain analytically.
Numerical derivatives also suffer from accuracy issues stemming from rounding or truncation errors.
Recent progress in automatic differentiation (AD) has.... \citep{Coogan:2022qxs}

\section{Differentiable Waveforms}
\label{sec:waveforms}

\subsection{Benchmarks}
\label{subsec:benchmarks}

\section{Applications}
\label{sec:applications}

\subsection{Effectualness Calculations}
\label{sec:effectualness}

\subsection{Fisher Analysis}
\label{subsec:fisher}

\subsection{Derivative Based Samplers - Hamiltonian Monte Carlo}
\label{subsec:hmc}


\bibliography{bib}

\end{document}
