% Define document class
\documentclass[twocolumn]{aastex631}
\usepackage{showyourwork}

% \usepackage[dvipsnames]{xcolor}
% \usepackage[colorlinks]{hyperref}
% \usepackage{colortbl}
\usepackage{multirow}
\definecolor{pyBlue}{RGB}{31, 119, 180}
\definecolor{pyRed}{RGB}{214, 39, 40}
\definecolor{pyGreen}{RGB}{44, 160, 44}
\definecolor{Burnt}{RGB}{203,96,21}

% Link setup
% \newcommand\myshade{80}
% \hypersetup{
%   linkcolor  = pyGreen!\myshade!black,
%   citecolor  = pyBlue!\myshade!black,
%   urlcolor   = pyRed!\myshade!black,
%   colorlinks = true
% }

\newcommand{\jax}{\texttt{JAX}\xspace}
\newcommand{\ripple}{\texttt{ripple}\xspace}
\newcommand{\lalsuite}{\texttt{lalsuite}\xspace}
\newcommand{\flowMC}{\texttt{flowMC}\xspace}
\newcommand{\CC}{C\nolinebreak\hspace{-.05em}\raisebox{.4ex}{\tiny\bf +}\nolinebreak\hspace{-.10em}\raisebox{.4ex}{\tiny\bf +}}

\newcommand{\TE}[1]{{\color{pyGreen} #1}}
\newcommand{\te}[1]{\textbf{\color{pyGreen}(TE: #1)}}
\definecolor{rb4}{HTML}{27408B}
\newcommand{\kw}[1]{{\color{rb4}[KW: #1 ]}}
\newcommand{\AZ}[1]{{\color{Burnt}[AZ: #1]}}

\definecolor{dfm}{HTML}{6C71C4}
\newcommand{\dfm}[1]{{\color{dfm}[DFM: #1]}}
\newcommand{\amc}[1]{{\color{red}[AC: #1]}}

% Begin!
\begin{document}

% Title
\title{\ripple: Differentiable and Hardware-Accelerated Waveforms for Gravitational Wave Data Analysis}

\newcommand{\JHU}{\affiliation{William H. Miller III Department of Physics and Astronomy, Johns Hopkins University, Baltimore, Maryland 21218, USA}} 
\newcommand{\OKC}{\affiliation{The Oskar Klein Centre, Department of Physics, Stockholm University, AlbaNova, SE-106 91 Stockholm, Sweden}} 
\newcommand{\NORDITA}{\affiliation{Nordic Institute for Theoretical Physics (NORDITA), 106 91 Stockholm, Sweden}}
\newcommand{\Ciela}{\affiliation{Ciela -- Computation and Astrophysical Data Analysis Institute, Montréal, Quebec, Canada}}
\newcommand{\UdeM}{\affiliation{Département de Physique, Université de Montréal, 1375 Avenue Thérèse-Lavoie-Roux, Montréal, QC H2V 0B3, Canada}} 
\newcommand{\Mila}{\affiliation{Mila -- Quebec AI Institute, 6666 St-Urbain, \#200, Montreal, QC, H2S 3H1}} 
\newcommand{\CGP}{\affiliation{Center for Gravitational Physics, University of Texas at Austin, Austin, TX 78712, USA}}
\newcommand{\CCA}{\affiliation{Center for Computational Astrophysics, Flatiron Institute, New York, NY 10010, USA}}
\newcommand{\CUHK}{\affiliation{Department of Physics, The Chinese University of Hong Kong, Shatin, N.T., Hong Kong}}

% Author list
\author{Thomas D.~P.~Edwards} \JHU \OKC \NORDITA
\author{Kaze W.~K.~Wong} \CCA
\author{Kelvin K.~H.~Lam} \CCA \CUHK

\author{Adam Coogan} \UdeM \Mila
\author{Daniel Foreman-Mackey} \CCA
\author{Maximiliano Isi} \CCA
\author{Aaron Zimmerman} \CGP

% Abstract with filler text
\begin{abstract}
    % Since their discovery in 2015, gravitational waves (GWs) have developed into a unique observational probe used throughout fundamental and astro-phyics.
    % As more detections continue, data analysis tasks are becoming increasingly computationally demanding.
    % \kw{I would cut everything before out.}
    We propose the use of automatic differentiation through the programming framework \jax for accelerating a variety of analysis tasks throughout gravitational wave (GW) science.
    Firstly, we demonstrate that complete waveforms which cover the inspiral, merger, and ringdown of binary black holes (i.e.~IMRPhenomD) can be written in \jax and demonstrate that the serial evaluation speed of the waveform (and its derivative) is similar to the \lalsuite implementation in \CC.
    Moreover, \jax allows for GPU-accelerated waveform calls which can be over five orders of magnitude faster than serial evaluation on a CPU.
    We then focus on three applications where efficient and differentiable waveforms are essential.
    Firstly, we demonstrate how gradient descent can be used to optimize the $\sim 200$ coefficients that are used to calibrate the waveform model. 
    In particular, we demonstrate that the typical \textit{match} with numerical relatively waveforms can be improved by more than 10\% without any additional overhead.
    Secondly, we show that Fisher forecasting calculations can be sped up by $\sim 100\times$ (on a CPU) with no loss in accuracy.
    This increased speed makes population forecasting substantially simpler.
    % , as discussed extensively in \citet{Iacovelli:2022bbs, Iacovelli:2022mbg}.
    Finally, we show that gradient-based samplers like Hamiltonian Monte Carlo lead to significantly reduced autocorrelation values when compared to traditional Monte Carlo methods.
    Since differentiable waveforms appear to have substantial advantages for a variety of tasks throughout GW science, we propose that waveform developers use \jax to build new waveforms moving forward.
    Our waveform code, \ripple, can be found at \href{https://github.com/tedwards2412/ripple}{github.com/tedwards2412/ripple}, and will continue to be updated with new waveforms as they are implemented.

\end{abstract}

\section{Introduction}
\label{sec:intro}

The discovery of gravitational waves (GWs)~\citep{LIGOScientific:2016aoc} from inspiraling and merging compact objects (COs) has revolutionized our understanding of both fundamental physics and astronomy~\citep[e.g.][]{LIGOScientific:2021djp,LIGOScientific:2021sio,LIGOScientific:2021psn}.
Although the data volumes from GW detectors such as Advanced LIGO~\citep{LIGOScientific:2014pky} and Virgo~\citep{VIRGO:2014yos} are relatively small, analyzing the data is a computationally demanding task.
In addition, this computational cost will substantially increase when next generation detectors come online~\citep{Maggiore:2019uih, Reitze:2019iox, Evans:2021gyd}.

The complexity begins even before data taking, since GW searches using the matched-filtering technique~\citep{Owen:1998dk, Owen:1995tm} require the generation of large banks of template waveforms.
% \sout{The complexity begins before data gathering where one is required to generate banks of template waveforms which will be used for a matched-filter search~\citep{Owen:1998dk, Owen:1995tm}. }
Once potential candidates are found, parameter estimation (PE) is performed to extract the detailed source properties of each event~\citep{Christensen:2022bxb, 2020MNRAS.493.3132S, Ashton:2018jfp, Romero-Shaw:2020owr, Veitch:2014wba, Biwer:2018osg}.
For binary black holes with non-aligned spins, this requires a Markov Chain Monte Carlo (MCMC) on an 15 dimensional parameter space.
More general examples can lead to a significant increase in dimension.\te{add citations}
% In the most minimal example of non-precessing binaries, this requires a Markov Chain Monte Carlo (MCMC) on an 11 dimensional parameter space for each event.
% \kw{This is not quite true. You can always marginalize over more parameters if you want, such as time of coalescence and phase. I would just mention the typical case of 15 parameters GR PE instead.}
% \AZ{Maybe modify; some other points are precessing PE requires 15 params, tidal prec 17 params, plus calibration marge and EoS inferences can incorporate even more}
Beyond this simple scenario, more complex waveform models with additional parameters may be used to test for deviations from General Relativity~\citep{Krishnendu:2021fga, LIGOScientific:2020tif, LIGOScientific:2016lio, Yunes:2016jcc, Arun:2006yw, Agathos:2013upa}.
% \AZ{Add PPE and maybe TIGER refs..?}
Finally, using the results of PE, population synthesis models constrain the progenitors systems from which the black holes we see merging today began their journey~\citep{LIGOScientific:2020kqk, LIGOScientific:2021psn, Wong:2022flg}.
\kw{Need more reference and some rewording}\te{added refs, happy?}
% All of these tasks require significant computation.
Overall, analyzing GWs therefore requires significant computation.
% \kw{This could be triggering to some people, since population analysis could be quite cheap}
In this paper, we will argue that differentiable waveforms (and more generally differentiable pipelines) can play a significant role in alleviating this computational demand.

Derivatives are ubiquitously useful throughout data analysis tasks.
% For instance, during PE, gradients are used to improve performance of the samplers 
% % \dfm{This isn't correct. In HMC the gradients don't have the behavior. I'd just say something like ``gradients are used to improve performance''} 
% (e.g.~in Hamiltonian Monte Carlo~\citep{2017arXiv170102434B} or Gradient Descent~\citep{2016arXiv160904747R}).
For instance, during PE, derivative information can be used to guide an optimizer towards higher-likelihood values (e.g.\ using gradient descent~\citep{2016arXiv160904747R}) or allow a sampler to rapidly explore parameter space (e.g.\ using Hamiltonian Monte Carlo~\citep{2011hmcm.book..113N,2017arXiv170102434B}).
Gradients are particularly valuable for high dimensional spaces.
Unfortunately, in the field of GW data analysis, analytic derivatives of the necessary quantities (such as the likelihood) have historically required a significant amount of work to obtain~\citep{Keppel:2013kia}.
With waveforms only increasing in complexity, calculating analytic derivatives will become ever more tricky.
% \kw{This needs some rewording. For example, I think in IMRPhenomD they do have the analytical derivatives coded up, at least for a part of the waveforms. Instead of difficult to have the derivatives, I would rather mention it is clumpsy to get the derivatives, especially given the model complexity is growing.}
Numerical derivatives also suffer from accuracy issues stemming from rounding or truncation errors.
However, recent progress in automatic differentiation (AD) has shown promise in allowing general, fast derivative calculations for gravitational waveforms, with applications to constructing template banks~\citep{Coogan:2022qxs} and computing the Fisher information for forecasting~\citep{Iacovelli:2022bbs, Iacovelli:2022mbg}.


Automatic differentiation is a family of methods used to compute machine-precision derivatives with little computational overhead. 
AD's recent ascendance is primarily driven by its use in machine learning, particularly for derivative computations of neural networks which use gradient descent during training.
The core idea of AD is that any mathematical function can be broken down into a small set of basic operations, each with a known differentiation rule.\footnote{
    Of course, non-differentiable functions exist and care must be taken when treating these special cases.
    }
The full derivative can then be constructed using the chain rule.
There are now a variety of AD implementations, most notably in deep learning frameworks such as \texttt{pytorch}~\citep{pytorch} and \texttt{tensorflow}~\citep{tensorflow2015-whitepaper}.
More general frameworks exist in \texttt{julia}~\citep{zygote, forwarddiff}, although \texttt{julia}'s limited use in GW analysis software precludes its general use.
Here we make use of \jax~\citep{jax2018github} due to its easy integration with \texttt{python} libraries, seamless support for running code on different hardware accelerators (e.g.\ graphical processing units) and its just-in-time (JIT) compiler, which can substantially accelerate code.

\AZ{Add refs here:}\te{Added a bunch, let me know if more are needed!}
There are a variety of gravitational waveforms currently used in analysis pipelines.
They are generally structured into different families, the most common of which are: the effective-one-body (EOB)~\citep{Damour:2008yg,Buonanno:2005xu, Buonanno:2000ef, Buonanno:1998gg, Damour:2000we}, the phenomenological inspiral-merger-ringerdown (IMRPhenom), and numerical relativity surrogate (NRsurrogate)~\citep{Varma:2018mmi,Varma:2019csw}.
%Of these, the IMRPhenom family is most well suited for an AD implementation in \jax.
Of these, the IMRPhenom family serves as a natural starting point for an AD implementation in \jax.
Models like the non-precessing IMRPhenomD~\citep{Khan:2015jqa} model studied here and the precessing, higher-mode model IMRPhenomXPHM~\citep{Pratten:2020fqn, Pratten:2020ceb} are written in the frequency domain using closed-form expressions.
This makes a
% \sout{In particular, their closed form expressions and fixed size frequency grids make a }
\jax implementation that complies with the constraints of JIT compilation simple.
NRsurrogate models, which interpolate directly over waveforms produced by numerical relativity simulations, are in principle also straightfoward to implement in \jax.
EOB waveforms on the other hand are produced by evolving the dynamics of an effective one body Hamiltonian, and are therefore more difficult to implement in \jax.
For EOB waveforms, frequency-domain reduced-order models may be a convenient target~\citep[e.g.][]{Cotesta:2020qhw}.
% to produce a waveform
%require dynamic frequency grids and are therefore tricky to implement in \jax.
%\AZ{What does this mean? I don't understand why EOB needs dynamic frequency grids. We should clarify, and we can also mention if \jax is well-suited to frequency-domain ROM implementations of EOB, and reference those.} 
%In addition, EOB waveforms use hypergeometric functions which have limited support in \jax.
%\AZ{rephrase this in a more positive fashion:}
%NRsurrogate waveforms have no fundamental issue but require the user to download relatively large data files, making their use cumbersome in practice.\te{Max/Kaze, please add to/change this if you think its incorrect}

In this paper we argue that differentiable waveforms will be a vital component for the future of GW data analysis.
% \AZ{I prefer to describe what is rather than what could be; can we say what elements exist, e.g.~
% we present \ripple... which includes \jax implementations of GW antenna patterns, the IMRPhenomD waveform, and...}
% In addition, we present \ripple, a small GW \texttt{python} package which can serve as a repository for differentiable implementations of some of the main waveforms currently used in LIGO and Virgo analyses. 
In addition, we present \ripple, a small GW \texttt{python} package which, at the time of writing, includes a \jax implementation of the IMRPhenomD waveform and will be continually updated with new waveforms as they are implemented.
% can serve as a repository for differentiable implementations of some of the main waveforms currently used in LIGO and Virgo analyses. 
The remainder of this paper is structured as follows. 
In Sec.~\ref{sec:waveforms} we discuss the differentiable IMRPhenomD waveform implemented in \ripple and perform some benchmarks to demonstrate its speed and accuracy. 
In Sec.~\ref{sec:applications} we discuss three distinct applications using differentiable waveforms. 
Firstly, we illustrate how the fit coefficients that form part of the IMRPhenom waveform models could be improved by high dimensional fitting enabled by a differentiable waveform. 
Secondly, we implement differentiable detector response functions and show that the speed of Fisher matrix calculations can be substantially accelerated using AD.
Finally, we run an illustrative injection example using Hamiltonian Monte Carlo to demonstrate that the autocorrelation of derivative samplers is substantially smaller than that of traditional MCMCs.
Practically, this translates into much faster PE with no sacrifice in accuracy.
The associated code can be found at \ripple~\citep{ripple}.

\section{Differentiable Waveforms}
\label{sec:waveforms}

A variety of waveform families have been developed to accurately model the GW emission from COs~\citep{Schmidtreview}. 
When the COs are relatively well separated, the dynamics of the system can be well approximated by a post-Newtonian expansion.
However, close to merger, numerical relatively (NR) simulations are required to accurately model the binary.
Unfortunately, these numerical simulations are computationally expensive and cannot be run in conjunction with data analysis.
Approximate, phenomenological waveforms have therefore been constructed to enable relatively fast waveform generation at sufficient accuracy.

As mentioned in the previous section, the three major waveform families that have been developed to date are:  EOB,  NRsurrogate, and IMRPhenom.
Note that unlike NRsurrogate, both the EOB and IMRPhenom waveforms must be calibrated to numerical relativity waveforms.
EOB waveforms require one to model the binary system using a Hamiltonian and are typically slow to evaluate, whereas IMRPhenom waveforms are constructed with simple closed-form expressions.
IMRPhenom waveforms are therefore ideally-suited for AD, especially using \jax. 
In this paper we focus on the aligned-spin, circular-orbit model IMRPhenomD~\citep{Husa:2015iqa, Khan:2015jqa}.

The implementation of IMRPhenomD in \lalsuite is in C, and therefore needs to be rewritten natively into \texttt{python} to be compatible with \jax.
We have re-written IMRPhenomD from scratch using a combination of pure \texttt{python} and \jax derivatives.
In addition, we have restructured the code for readability and evaluation speed as well as exposing the internal fitting coefficients to the user (which we will use later in Sec.~\ref{sec:applications}).

To demonstrate our implementation of IMRPhenomD is faithful to the \lalsuite implementation, we start by defining the noise weighted inner product:
\begin{equation}
    \label{eq:inner_prod}
    \left(h_1|h_2\right) \equiv 4 \, \mathrm{Re} \int^{\infty}_{0} \mathrm{d} f \, \frac{ h^*_1(f) h_2(f)}{S_n(f)}\, ,
\end{equation}
where $S_n$ is the (one-sided) noise power spectral density (PSD) and \AZ{$h_1$ and $h_2$ are the frequency domain waveforms which are to be compared.}
\sout{ evaluated at the intrinsic parameter point $\theta_1$.}
We can then normalize the inner product through
\begin{equation}
    \left[h_1|h_2\right] = \frac{\left(h_1|h_2\right)}{\sqrt{\left(h_1|h_1\right)\left(h_2|h_2\right)}}\, .
\end{equation}
Now we are ready to define the \textit{match} which is given by 
\begin{equation}
    \mathrm{m}(h_1|h_2) \equiv \max_{\Delta t_c,\, \Delta \phi_c} \left[h_1 \, \middle| \, h_2 \right]\,
\end{equation}
where \AZ{$\Delta t_c$ and $\Delta \phi_c$ are, respectively, the differences in time and phase of coalescence between the two waveforms.}

Since the match is a measure of the difference between two waveforms, we can use it to demonstrate that the implementation of IMRPhenomD in \ripple accurately matches the \lalsuite implementation.
\AZ{Say what PSD was used and reference it.}
This is shown in Fig.~\ref{fig:match}, where we have calculated the $\mathrm{m}(h_1|h_2)$ across the entire parameter space.\footnote{
    Specifically, we use $10^4$ points \AZ{varying component masses $m_1$, $m_2$ and spin parameters $\chi_1$, $\chi_2$} in the ranges $m_{1,2} = (1,100)\,M_{\odot}$ and $\chi_{1,2} = (-1,1)$. In addition, we evaulated the waveforms on a frequency grid from $32\,$Hz to $1024\,$Hz with frequency spacing $\Delta f = 0.0125 \,$Hz.
}
\AZ{Here, $h_1$ corresponds to the \ripple waveform implementation and $h_2$ is the \lalsuite implementation, evaluated at the same point in parameter space.} 
From Fig.~\ref{fig:match}, it is clear that the \ripple waveform matches with the \lalsuite waveform close to numerical precision across the entire parameter space.
\AZ{strictly $=0$? That seems impossible. Fmla has wrong labels for $h$'s}
In particular, the gray points at low masses are where $1-\mathrm{m}(h_1|h2)=0$ causing points to not appear in the log scale colorbar.
% In particular, the blue points indicate parameter points where the $1-\mathrm{m}(h_{\theta_1}|h^{*}_{\theta_1})=0$ and would therefore not appear in the log scale colorbar.
\begin{figure}[t]
    \script{random_matches.py}
    \begin{centering}
        \includegraphics[width=\linewidth]{figures/random_matches.pdf}
        \caption{
            Match between the \ripple and \lalsuite implementations of the IMRPhenomD waveform as a function of total mass and effective spin. 
            % The lower density of points at low masses is caused by the match being exactly 1.0 and therefore not appearing in the log scale. 
            The gray points are where the match is exactly 1.0 and therefore would not appear in the log scale.
            It is clear that the \ripple waveform matches the \lalsuite implementation many of orders of magnitude more than necessary for all data analysis tasks across the entire parameter space.
            \dfm{I think this figure could use some work. Maybe the points could be smaller to make them easier to resolve. And it could be interesting to clip the mismatch to something like $10^{-16}$ so that we don't lose the points with zero mismatch. Right now it looks less impressive!}
        }
        \label{fig:match}
    \end{centering}
\end{figure}

Note that in General Relativity (GR) the total mass $M = m_1 + m_2$ of \AZ{a binary black hole simply serves as an overall scale for the system, so that the frequency evolution can be trivially rescaled.
The total mass only impacts the match because the chosen PSD and frequency limits fix a reference scale.}
\sout{the system is simply an overall scaling of the waveforms} 
\sout{frequency evolution and is therefore not a} \sout{truly independent parameter.}
If we instead use a flat PSD in Eq.~\ref{eq:inner_prod} and \AZ{rescale the frequency grid to units of $Mf$}, \sout{correctly,} all dependence with $M$ seen in Fig.~\ref{fig:match} vanishes. 
\AZ{Revisit this after defining PSD above:} 
This figure instead illustrates a more realistic PSD where at low masses the waveform signal-to-noise is dominated by the inspiral but at high masses it is dominated by the merger.

There remains some slight deviation between the two waveform implementations at high total mass. 
This is partly due to the fact that cubic interpolators, which are used within IMRPhenomD to calculate the ringdown and damping frequencies, are not currently supported in \jax. \dfm{What dimension of interpolation do you need? Maybe this is something we could implement...}
Instead, we initially use \texttt{scipy}'s cubic interpolator to create a fine grid of $5\times10^5$ values, which we then linearly interpolate during waveform generation.
Unfortunately, we cannot make the initial cubic interpolation arbitrarily fine as this would add additional computational overhead when loading the data during waveform evaluation.
% The remaining error in the waveform at high masses is therefore primarily down to the coarseness in this interpolation.
Note however, that the differences are well below the accuracy requirements of the waveforms and will have no noticeable for realistic data analysis tasks.

For maximum utility, a waveform needs to be fast to evaluate.
Fortunately, the IMRPhenom waveforms are constructed from simple closed-form expressions which are computationally efficient.
Since the \lalsuite implementation of IMRPhenomD is written in \texttt{C}, the evaluation of the waveform itself is faster than \ripple. \dfm{I'd soften this statement a little bit. In my own work, I often find that JAX implementations are just as fast or faster than C implementations, so I wouldn't state it that it's obvious that lalsuite should be faster! Where are these benchmarks implemented? Would it be easy for me to try them out?}
Benchmarking on a MacBook Pro with an M1 Max Apple Silicon processor, we find that a single waveform evaluation takes $\sim 2-5\,\mathrm{ms}$ for \lalsuite (interfaced with python) and $\sim 5-8\,\mathrm{ms}$ for \ripple.\footnote{
    For this benchmark we used $16\,\mathrm{Hz}$ and $1024\,\mathrm{Hz}$ for the lowest and highest frequencies respectively. 
    In addition, we used a frequency spacing of $0.0125\,\mathrm{Hz}$. 
    We performed this benchmark by evaluating the waveform $5000$ times and taking the average evaluation time.
}
Although slower on a CPU benchmark, a key advantage of using \jax is its native ability to run on a GPU.
Waveform calls can therefore be evaluated in parallel, massively improving performance.
Performing the same benchmark as above on a Google Colab notebook with a freely-provided NVIDIA Tesla T4 GPU (far from the current state-of-the-art), we find that on average waveform evaluations take $\sim \sout{0.00005} 5 \times 10^{-5}\mathrm{ms}$, nearly five orders of magnitude faster.
\kw{We should add some more detail related to this comparison. We should be clear that each individual waveform sample evaluation doesn't get us the performance by putting it on GPU (Jitting helps with that, not the GPU part). The GPU comes into play by allowing parallel evaluation of waveform sample, so for a tiny amount of time/frequency samples, we should expect GPU to lose to a CPU, but on any reasonable amount of time/frequency samples, GPU win a lot.}
\amc{mention \texttt{vmap} and/or \texttt{pmap} -- super trivial to parallelize! More work in C.}
Generalizing \lalsuite waveforms to run on a GPU would be a significant undertaking.

The primary argument of this paper is that waveform derivatives will also be highly valuable to data analysis tasks. 
AD provides two big advantages when it comes to evaluating derivatives compared to numerical differentiation.
First, the accuracy of derivatives from AD are significantly more stable than finite-difference methods.
In particular, finite differences suffer from both rounding and truncation errors, meaning that the user is required to \textit{tune} the width over which the difference is taken.
On the other hand, AD produces machine-precision derivatives with no tuning.
Second, AD scales favorably with the dimensionality of the function.
In particular, for every input dimension, $D$, added one would need to evaluate the function at least $2D$ times to calculate finite difference derivatives for all input parameters.
For reverse-mode AD, one only needs two function calls to evaluate the derivative of all input parameters, regardless of dimension.\footnote{
    Note that although the number of function calls is small, reverse-mode AD does add memory overhead.
    We've not found this to be limiting in any of the situations tested so far.
}
\kw{Maybe a bit more background on AD, the way we phrase it here sounds a bit too magical too me now.}
Since the parameter space of GWs in GR has $\mathcal{O}(10)$ dimensions, the speed of derivative evaluation is less crucial than the stability.
However, this might change for waveforms in beyond-GR models, \AZ{waveforms involving equation of state parameters for neutron stars, and models which account for calibration and waveform uncertainties.
In these cases} many more parameters \AZ{can be} added.

Overall, we have demonstrated that the IMRPhenom waveform family is ideally suited for AD.
Moreover, we have shown that our implementation of IMRPhenomD in \ripple is accurate and quick to evaluate, especially when hardware acceleration is available.
In the next section, we will discuss a variety of potential use cases of differentiable waveforms.

% Emphasise the stability of gradients. $Done$
% Not just a faster way of evaluating PhenomD. $Done$
% Change plot to be q vs chieff. $Done$
% Note that lalsuite cannot be made massively parallel on GPUs. $Done$
% Don't need a full section on Fisher, but make sure to mention. People love Fisher.
% Use acceptance rate as a good explanation of speed ups
% Wall time a messy comparison. Based too much off of hardware
% Emphasise that gradients are beneficial

\section{Applications}
\label{sec:applications}

Here, we illustrate how three core tasks in GW science can be substantially improved through the use of differentiable waveforms.
In the paper we primarily look at toy examples, leaving a more careful analyses to future work. 
The three tasks discussed here cover a wide range of GW science, starting with waveform development all the way to Fisher forecasting and PE.

\subsection{Fine-Tuning Waveform Coefficients}
\label{subsec:coeffs}

\te{We use in total 11 waveforms for the optimization. Original PhenomD uses 19 @Kelvin add details of where to find this.}
\te{Fig 2, plot was made after optimizing for ALL waveforms}
\te{536 waveforms are subset of the SXS catalogue that are not precessing or eccentric. @Kelvin add details of where to find these things}
\te{None of the 536 are part of the training set}
\te{Used ALIGO design sensitivity for PSD}

\AZ{Broad comments: need to cite which catalog the waveforms came from if any.
The number of waveforms used seems much larger than the number originally used
to calibrate PhenomD. 
If that is true, we can't make a big deal about having a better mismatch compared to
the original version, because of course optimizing over another set will improve 
performance.
Say how many waveforms we use to optimize, how many we compare to, where they came
from.}

Having an accurate waveform model is essential for many data analysis tasks in
GW, such as searching for signals and estimating source parameters given data.
While waveforms generated using numerical relativity simulations are \sout{commonly
regarded as the most accurate,} 
\AZ{in principle the highest fidelity signal model,}
they are too computationally expensive to be used
in any practical data analysis tasks. To create waveform models that can be used
in data analysis, there are a number of groups creating ``approximants'' of the
full numerical relativity waveforms, which are essentially simpler ansatzes that
\sout{are only approximations but} can be evaluated much faster \AZ{and in regions not
covered by numerical simulations, such as extremes of parameter space or earlier 
phases of the binary inspiral}. 
\amc{redundant -- remove or merge with earlier material.}

%\sout{Every waveform approximant has some}
\AZ{Waveform approximants generally have free coefficients which are calibrated
to NR waveforms to achieve high accuracy.}
\sout{ fitting coefficients that are tuned during
the process of calibrating the ansatz to NR waveforms.} In the case of
IMRPhenomD, there are 209 fitting coefficients used to capture the separate
behavior of the amplitude and phase as a function of the mass ratio and spins.
\sout{The accuracy of fitting coefficients determines how well the approximant
reproduces the NR waveform. 
The}
\AZ{Any} inaccuracy in obtaining the fitting coefficients
means misrepresentation of the NR waveform, which can translate to systematic
error in downstream data analysis task. For example, sufficiently large
systematic errors in the waveform would cause the recovered source parameters to be biased
in the case of PE.

Previously in the construction of IMRPhenomD \citep{Khan:2015jqa}, waveform
coefficients were fitted in segments with amplitude and phase separated.
Furthermore, IMRPhenomD is divided in three fitting segments, inspiral, merger and ringdown.
Each of these segments has their own set of fitting coefficients.
After obtaining the fitting coefficient for individual segments, they are then ``stitched'' together such that they are continuous in the first derivative.
The process of stitching introduces some additional
inaccuracy in the waveform model, as the connections would affect the originally
fitted segments.

The coefficients of the current IMRPhenomD implementation are tuned in subsets
of parameters instead of all together. This means the tuning process ignores the
correlation between different subsets of parameters, so the best-fit
solution obtained by fitting subsets of parameters may not be the global
optimum compared to fitting all coefficients at once. By jointly
fitting all coefficients, the correlation in fitting individual and connecting
different segments can be accounted for during the optimization. Therefore, there is
potential improvement of the accuracy by fitting all coefficients at once.

\AZ{Maybe comment on optimization method used for EOB waveforms?
I think they do interesting Monte Carlo etc. Check what method is used 
for IMRPhenomXPHM?}

In general, optimization problems in a high dimensional space benefit from
having access to the gradient of the objective function. Since we can
differentiate through the entire waveform model against the fitting parameters,
one can use gradient descent to find the optimal fitting parameters all at once
instead of fitting them in subsets.




The first step in fitting all coefficients is to define a loss function that
measures the goodness-of-fit of the optimization procedure, which we choose it
to be the mismatch between the NR waveform and the approximant waveform:
\begin{equation}
    \label{eq:mismatch}
	\mathcal{M}(\lambda)=1-\mathrm{m}(h_{\mathrm{IMR}}(\lambda)|h_{\mathrm{NR}}(\lambda)),
\end{equation}
where $\lambda$ is a vector of the fitting coefficients, $h_{\mathrm{IMR}}$ is
the waveform generated by the IMRPhenomD and $h_{\mathrm{NR}}$ is the waveform
generated by the numerical relativity simulation. 
Given the loss function, we use gradient descent to find the optimal fitting
coefficients.
\begin{equation}
	\lambda\leftarrow\lambda-\alpha\nabla\mathcal{M}, 
\end{equation}
where $\alpha$ is the learning rate. We set $\alpha$ to be a very small number, i.e. $10^{-6}$, since the initial guess already lies close to the desired optimum. We terminate the program after 30000 steps, in which $\lambda$ has converged to the optimum.  \dfm{We can't possibly know that it has converged to ``the optimum''. I'd sat just leave it at ``We terminate the program after 30000 steps.''}
\amc{say something to explain why this is working. I.e.: does the loss initially drop and then plateau?}
% Add description of the optimization procedure.

As a demonstration, Fig. \ref{fig:loss_compare} shows the relative error of
original and optimized waveform at different frequencies. In this optimization
procedure, we took NR waveforms $h_{\mathrm{NR}}$ stated in \citep{Khan:2015jqa}
as training waveforms to fit waveform coefficients. \amc{how many are there?}
We can see the error of optimized waveform is lower than that of the original waveform for most of the domain.
In particular, the error in the early inspiral region is decreased by half while other regions also show good improvement in accuracy. \amc{so this is a waveform in the training set? And I don't follow exactly what you did. Is this just fitting a single waveform?}

We can generalize the fitting procedure described above to a collection of waveforms.
In that case, the loss function is defined as the average of the mismatch of
individual waveforms, given by
\begin{equation}
	\mathcal{L}=\frac{1}{N}\sum_{i=1}^{N}\mathcal{M}_i,
    \label{eq:multi_loss}
\end{equation}
where $\mathcal{M}_i$ is the mismatch of individual training waveforms and $N$
is the total number of training waveforms used in optimization. This
optimization is more difficult since we are applying the same set of
coefficients to waveforms with different intrinsic parameters, such as mass
ratio and spins. From Eq. \ref{eq:multi_loss}, we can see the averaging between
waveforms with different intrinsic parameters implies there are trade-offs in
performance in different regions of the intrinsic parameter space.
\amc{this means your fit generally will depend on how the training waveforms are distributed over parameter space.}

\begin{figure}[t]
	\script{loss_compare.py}
	\begin{centering}
		\includegraphics[width=\linewidth]{figures/loss_compare.pdf}
		\caption{
			Relative error between the numerical relatively and IMRPhenomD waveform amplitudes. 
            The blue line shows the relative error of the IMRPhenomD model with the original model coefficients where as the orange line uses the IMRPhenomD model after optimization using gradient descent.
            Interestingly, the error is reduced across the frequency range, demonstrating that high dimensional optimization is useful for all parts of the waveform.
		}
		\label{fig:loss_compare}
	\end{centering}
\end{figure}

\begin{figure}[t]
	\script{mismatch_hist.py}
	\begin{centering}
		\includegraphics[width=\linewidth]{figures/mismatch_hist.pdf}
		\caption{
			Distribution of 536 log mismatches (see Eq.~\ref{eq:mismatch}) for the original (blue) and optimized (orange) IMRPhenomD. 
            Overall, the optimized distribution shifts to lower mismatches and therefore a better waveform model. 
            More quantitatively, the mean mismatch (shown as vertical dashed lines) is reduced by $\sim11$\%.
		}
		\label{fig:mismatch_hist}
	\end{centering}
\end{figure}
 
In Fig.~\ref{fig:mismatch_hist}, we show the distribution of log mismatches.
One can see the distribution of mismatch after optimization has smaller mismatch compared to before optimization,
showing the optimization improves the accuracy of the waveform model.
% However, it is not improving for every waveforms. Say why briefly, and say there is another paper coming out.
While many waveforms improved their accuracy, some retain similar mismatches. 
\AZ{I don't understand this claim. The spin degeneracy is present in the physics of the waveform; $\chi_{\rm eff}$ is just one way of parametrizing the spins that tries to reduce the degeneracy.}
This phenomenon is mainly due to the reduced spin approximation used in IMRPhenomD, where spin degeneracy is introduced, hence restricting further improvements. 
In an upcoming paper, we will give a detailed discussion of how such an approximation scheme affects the optimization procedure. 
Nevertheless, the peak of the original waveform distribution has moved to the left end of the distribution, indicating that \sout{the} \AZ{our AD-assisted} optimization procedure \AZ{provides an improved implementation of the model}. 
\sout{Therefore, the new set of waveform coefficients can produce better waveforms for GW analyses. }

\subsection{Fisher Forecasting}
\label{subsec:fisher}

\begin{figure}[t!]
    \script{sky_localization.py}
    \centering
    \includegraphics[width=\linewidth]{figures/sky_localization.pdf}
    \caption{
        Distribution of Fisher information sky localization error for a population of nearby binaries.
        Automatic differentiation (AD) was used to compute the derivatives in Eq.~\ref{eq:fisher}.
        As emphasized in the text, after JIT compilation, each error calculation was around 100 times faster than a similar forecasting code \texttt{GWbench}~\citep{Borhanian:2020ypi}.
    }
    \label{fig:sky_localization}
\end{figure}

Forecasting the sensitivity of future experiments is an essential task in GW science. \kw{LOL I won't say it is essential.}
Due to its theoretical simplicity and evaluation speed, the Fisher matrix formalism is commonly deployed to estimate how well a binary system's parameters can be measured.
The Fisher matrix approach is built around the assumption of a Gaussian likelihood.
Although in practice this assumption is often violated for realistic detector noise, the results obtained using a Fisher analysis can provide quick and useful diagnostics in evaluating sensitivities for a variety of models and detector configurations.

Computing the Fisher matrix requires one to evaluate derivatives of the likelihood, which in turn involves derivatives of the waveform model and detector projection functions.
AD is therefore perfectly suited for computing Fisher matrices accurately and efficiently. 
Forecasting with Fisher matrices for third generation detectors has already been extensively explored in \citep{Iacovelli:2022bbs, Iacovelli:2022mbg}.
Here we purely want to illustrate the simplicity and speed of AD for forecasting rather than providing new physics insights.
We therefore consider a simple, three-detector setup corresponding to the two LIGO detectors in addition to Virgo.

The Fisher information matrix for a single detector is typically given by \amc{I'm guessing in practice you compute a Hessian, so maybe write that out?}
\begin{equation}
    \label{eq:fisher}
    \mathcal{I}^{k}_{ij} = (\partial_i h^k | \partial_j h^k) \, ,
\end{equation}
where $k$ indicates the detector, $\partial_i = \partial/\partial \theta_i$, and $h$ is the strain measured by the detector which is given by,
\begin{equation}
    h^k(\theta) = F_+(\phi) h^k_{+}(\Xi) + F_\times(\phi) h^k_{\times}(\Xi) \, .
\end{equation}
Note that here we have separated out the extrinsic ($\phi$) and intrinsic ($\Xi$) variables as well as introducing the detector projection functions for the plus and cross polarizations as $F^k_+$ and $F^k_\times$ respectively.
Since we are considering a three detector setup we simply add the Fisher matrices from the individual detectors to get the combined Fisher matrix:
\begin{equation}
    \mathcal{I}_{ij} =  \mathcal{I}^{\mathrm{Hanford}}_{ij} + \mathcal{I}^{\mathrm{Livingston}}_{ij} + \mathcal{I}^{\mathrm{Virgo}}_{ij}   \, .
\end{equation}
Finally, we invert the Fisher matrix to calculate the covariance matrix,
\AZ{which proves forecasted measurement errors and parameter covariances for a signal with parameters $\theta$ observed by the given detector network}.
\AZ{Again we need to state and cite the PSDs we use for the three detectors.}

% \begin{table*}[ht!]
%     \begin{center}
%     \begin{tabular}{c|c|c|c|c|c|c|c|c}
%     \hline\hline
%     \hfil $\boldsymbol{m_1, m_2}$ &
%     \hfil $\boldsymbol{\chi_1, \chi_2}$ &
%     \hfil $\boldsymbol{D}$ &
%     \hfil $\boldsymbol{t_c}$ &
%     \hfil $\boldsymbol{\phi_c}$ &
%     \hfil $\boldsymbol{\iota}$ &
%     \hfil $\boldsymbol{\psi}$ &
%     \hfil $\boldsymbol{RA}$ &
%     \hfil $\boldsymbol{DEC}$
%     \\
%     \hline\hline
%     \hfil [0.1, 3] &
%     \hfil [0, 1] &
%     \hfil [0, 1] &
%     \hfil [0.7, 1.3] &
%     \hfil [-3, 0] &
%     \hfil [0, 3] &
%     \hfil [-3, 3] &
%     \hfil [-3, 3] &
%     \hfil [-3, 3]
%     \\
%     \hline\hline
%     \end{tabular}
%     \end{center}
%     \caption{caption here}
%     \label{tab:priors}
% \end{table*}

\begin{table}[t]
    % \begin{center}
    \centering
    \begin{tabular}{c|c}
    \hline\hline
    $m_1, m_2$ & $\boldsymbol{U}[1.0, 50]\, \mathrm{M}_\odot$ \\ \hline
    $\chi_1, \chi_2$ & [-0.99, 0.99] \\ \hline
    $D$ & [500, 1000]\, Mpc \\ \hline
    $t_c$ & 0.0 \\ \hline
    $\phi_c$ & 0.0 \\ \hline
    Inclincation Angle, $\iota$ & $\boldsymbol{U}[0, 1]$ \\ \hline
    Polarization Angle, $\psi$ & $\boldsymbol{U}[0, 1]$ \\ \hline
    Right Ascension, $\delta$ & $\boldsymbol{U}[0, 1]$ \\ \hline
    Declination, $\alpha$ & $\boldsymbol{U}[0, 1]$ \\ 
    \hline\hline
    \end{tabular}
    % \end{center}
    \caption{Priors for the 11 dimensional parameter space used for the Fisher forecasting population analysis in Sec.~\ref{subsec:fisher}. 
    $\boldsymbol{U}$ indicates a uniform distribution between the two variables in the brackets.
    \AZ{$\cos \iota \in (-1,1)$, $\psi \in (0,2 \pi)$ (or $\pi$), RA is $\alpha \in (0, 2\pi)$, Dec is $\delta$, let $\vartheta = \pi/2 - \delta$, then $\cos (\vartheta) \in (-1,1)$ }
    }
    \label{tab:priors}
\end{table}

To illustrate the computational speed of computing Fisher matrices with AD, we consider a population of binaries and compute the sky localization error following Eq.~(28) in \cite{Iacovelli:2022bbs, Iacovelli:2022mbg}.
Since the Fisher matrix approach is known to have both theoretical issues as well as numerical instabilities for low signal-to-noise events, we restrict our population to only nearby systems.
A full list of the distributions used to generate the various parameters in our population are given in Tab.~\ref{tab:priors}.
The resulting population produces binaries with signal-to-noise ratios ranging from $\mathcal{O}(10-10^2)$.

The distribution of sky localization errors from a population of $10^3$ binaries can be seen in Fig.~\ref{fig:sky_localization}.
We have verified that our errors agree with a separate dedicated Fisher forecasting code~\citep{Borhanian:2020ypi} to within {\color{red}{X}} percent.
This demonstrates that AD can be used to accurately produce population-level forecasts.

Moreover, each error calculation (including computing the Fisher matrices for each detector and the inversion process) is substantially faster.
In particular, we find that after compilation each error calculation takes approximately half a second on a single computing core.
\texttt{GWbench}~\citep{Borhanian:2020ypi}, on the other hand, takes $\mathcal{O}$(minutes) for each Fisher calculation using the same detector setup and frequency grid.
This factor of over 100 speed up is substantial considering the fact that a single core evaluation of the \ripple waveform is slower than the \lalsuite call \AZ{used by \texttt{GWbench}}.
\kw{Want to pull a number on the total time it took on generating the population result? I imagine this should turn the theorists on and send them into a fishing frenzy.}
As discussed above, performance can be further improved by utilizing hardware acceleration such as parallel GPU processing.
AD therefore represents a fast and accurate way of performing population level analyses, and should be utilized for testing the capabilities of next generation detectors.

\subsection{Derivative Based Samplers - Hamiltonian Monte Carlo}
\label{subsec:hmc}

\begin{figure*}[t]
    \script{corner_plot.py}
    \centering
    \includegraphics[width=\linewidth]{figures/PE.pdf}
    \caption{
        Corner plot for the posteriors from an HMC (see text for details) on simulated noise with injected signal. 
        Orange lines indicate the true values of the injection. 
        Although not fully converged, it is clear that we find posteriors consistent with the injected parameters. \amc{could apply a mild Gaussian kernel to smooth this for the sake of presentation. Are the parameter ranges due to your priors? How did you handle the boundaries in the HMC?}
    }
    \label{fig:corner}
\end{figure*}

\amc{most acronyms were already introduced.}
After the search algorithms have constructed a list of confidently detected binaries, the next step is to sample from the posterior of each sources parameters - so called PE.
To do this, one typically uses a Markov Chain Monte Carlo (MCMC) or nested sampler~\citep{Skilling2004, multinest, dynesty}.
Although robust, both MCMC and nested sampling are slow to converge and are known to perform poorly in high dimensional parameter spaces.
For example, sampling the 15 dimensional parameter space for a BBH system can take $\mathcal{O}(10)$ hours, while BNS systems can take up to weeks.
Dedicated fast samplers have been designed to get approximate posteriors on the sky localization to facilitate follow-up electromagnetic observations \citep[e.g.~\texttt{BAYESTAR}][]{Singer:2015ema}.
\AZ{We should be careful here. Many new algorithms have sped this a lot. Cogwheel can do fast PE, DINGO is a normalizing flow code that is very fast when trained. There are likelihood acceleration methods like ROQ, multibanding, relative binning/heterodyning that accelerate the process compared to the numbers quoted above.}
Nevertheless, these do not present the whole picture; fast, general PE therefore remains a key aim of GW data analysis

A primary issue with both MCMC and nested sampling is that neither utilizes information about the likelihood's derivative and must therefore randomly walk towards areas of highest likelihood.
Derivative based samplers, on the other hand, have been shown to extrapolate well to higher dimensions although they sometimes come with their own drawbacks.
Here we simply aim to demonstrate the utility of a derivative based sampler and their efficiency on a small test problem.
% In particular, we will show that proposals made using an Hamiltonian Monte Carlo (MC) sampler are accepted at a much higher rate than a traditional MCMC algorithm.
In particular, we will show that the autocorrelation of a Hamiltonian Monte Carlo (HMC) sampler is significantly lower than a traditional MCMC algorithm.
\AZ{Add citations to other HMC papers in GW land}

For our basic example we perform an injection recovery test on a seven dimensional parameter space with the same three detector network setup considered in \S~\ref{subsec:fisher}.
\AZ{As before, say what PSD and cite.}
More specifically, we generate Gaussian noise consistent with the measured PSDs for each detector and then inject a BBH signal with parameters: chirp mass $M_c = 23.82\,\mathrm{M_\odot}$,  symmetric mass ratio $\eta = 0.248$, primary spin parameter $\chi_1=0.3$, secondary spin parameter \AZ{$\chi_2=-0.4$}, luminosity distance $D = 1.6\,\mathrm{Gpc}$, coalescence time $t_c = 0.0$, and coalescence phase $\phi_c = 0.0$.\footnote{
    The remaining parameters are set to $\pi/3$. \AZ{Can we say what they are? $\iota$, $\psi$, $\alpha$, $\delta$?}
}
Using a standard Gaussian likelihood, we then run the HMC sampler implemented in \flowMC for \sout{200000} $2\times10^5$ steps and the Gaussian random walk (GRW) sampler for \sout{1500000} $1.5\times10^6$ steps (each with four randomly initialized independent chains). 
\AZ{What implementation of GRW? Can we cite something?}
\amc{briefly summarize and give refs for flowMC and GRW. Is GRW Metropolis-Hastings with a Gaussian transition kernel? How'd you tune this? Why not use NUTS for HMC?}
The number of steps and and mass matrix used for each example was hand tuned to give good performance for the specific sampler.
\amc{not a huge deal, but why hand tune? There are automated ways to do this in e.g.\ numpyro with a warm-up adaptation phase.}
The additional steps for the GRW sampler were required to achieve a similarly converged posterior.
\amc{Why not plot its posterior?}

In Fig.~\ref{fig:corner}, the grey contours show the posterior recovered using the best chain (i.e. one that reached the highest log-likelihood values).
The orange shows the true parameters of the injected signal.
From the one dimensional histograms along the diagonal, it is clear that we consistently recover all seven parameters apart from $\phi_c$.
This is expected since the injected binary is relatively nearby.
\AZ{Can we give the SNR?}

\begin{figure}[t]
	\script{autocorrelation.py}
	\begin{centering}
		\includegraphics[width=\linewidth]{figures/autocorrelation.pdf}
		\caption{
			Autocorrelation as a function of lag for HMC and GRW on the same simulated data as discussed in Fig.~\ref{fig:corner}.
            The smaller autocorrelation of the HMC leads to a larger number of independent samples and therefore a faster converging Monte Carlo.
		}
		\label{fig:autocorrelation}
	\end{centering}
\end{figure}

% To illustrate the benefit of a gradient based sampler (such as HMC), we run the same sampling as above but instead use a Gaussian random walk (GRW) sampler. 
% We sample for the same number of steps as used in the HMC with identical initializations to explore acceptance rates of new points.
% For the HMC example, find that the mean local acceptance rate across the 15 chains is around $\sim0.28$, where as the local acceptance rate for the equivalent GRW sampler is XXX.
% Since we are running the chains from the same initialization, this enhanced acceptance for HMC can be mainly associated with use of gradient information to guide sampling.
Although further steps would be required to achieve a fully-converged posterior, these chains are sufficient to show the increased efficiency associated with HMC. 
To further illustrate this, in Fig.~\ref{fig:autocorrelation} we plot the autocorrelation as a function of lag for both the HMC and GRW samplers.
The HMC autocorrelation is substantially lower than that of the GRW.
We therefore expect gradient based samplers to converge significantly faster than typical samplers, especially in higher dimensions.
In addition, we found that the effictive number of samples 
\amc{provide ref?} 
(a measure of the number of indepedent samples) is between 2 and 7 times larger for HMC across the different dimensions of the parameter space.

In a follow up paper we will demonstrate that minute scale PE can be achieved by combining normalizing flows~\citep{2022arXiv221106397W, Gabrie:2021tlu}, GPU acceleration, and a derivative based sampler~\citep{PEpaper}. 
We therefore expect \jax waveforms to be \sout{crucial} \AZ{highly beneficial} to future PE efforts in GW astronomy, especially for low-latency pipelines.
\kw{Not just low latency, but also high dimensional analysis, such as testing GR and lensing.}

\section{Discussion and Conclusion}
\label{subsec:discussion}

In this paper we introduced and discussed the various benefits of differentiable waveforms in \jax for GW data analysis.
First, we demonstrated the speed and accuracy of our implementation of the aligned spin IMRPhenomD waveform.
In particular, we showed that it matches the \lalsuite implementation to near machine precision and can be easily parallelized on a GPU.
Parallelization on a GPU provides substantial speed increases; on a free Google Colab GPU we found that waveform evaluations are nearly 5 orders of magnitude faster than serial CPU evaluations.
Second, we discussed three data analysis tasks which can all be substantially improved by utilizing derivative information of the waveform.
Although we primarily discuss toy examples in this paper, each can be extended to the full data analysis task, some of which will be shown in upcoming papers~\citep{PEpaper}.
Differentiable waveforms therefore represent a crucial tool to perform efficient GW science.

In this paper, we have primarily focussed on the IMRPhenom family of waveforms as their closed form expression is perfectly suited for implementation in \jax.
Two other waveform families are commonly used in GW data analysis: the effective-one-body (EOB) and numerical relativity surrogate (NRSurrogate). 
Some progress has been made towards the construction of a differentiable NRsurrogate, but currently it seems difficult to implement EOB waveforms in \jax.
In particular, the evolution of the Hamiltonian required to evaluate an EOB waveform is both inherently slow to differentiate and difficult to implement in \jax. 
\sout{which requires fixed length arrays for its JIT compilation.}
Since EOB methods are used to produce state-of-the-art waveforms for many applications,
\sout{are often regarded as the state-of-the-art,} 
more work is required to see if a fast, differentiable implementation is possible.

Currently the biggest constraint to adopting differentiable waveforms is the need to rewrite the most commonly used waveforms into \jax (or pure python).
In order to showcase the benefits of differentiable waveforms as quickly as possible, at the time of writing, we have only implemented an aligned spin GW model (IMRPhenomD).
We plan on adding a variety of different waveforms to \ripple in the near future with the primary goal of reaching a \jax version of a fully precessing, higher order mode waveform such as IMRPhenomXPHM~\citep{Pratten:2020ceb}.
Ideally, future waveforms should be implemented either in pure python or using \jax directly.
\amc{if they're pure python, some translation will be required due to jax's restrictions.}
This would ensure that the community can easily utilize differentiability and hardware acceleration in the future.


\section{Acknowledgments}
This material is based upon work supported by NSF's LIGO Laboratory which is a major facility fully funded by the National Science Foundation.
This work was supported by collaborative visits funded by the Cosmology and Astroparticle Student and Postdoc Exchange Network (CASPEN). 
T.E.\ is supported by the Horizon Postdoctoral Fellowship.
A.Z.~is supported by NSF Grant PHY-2207594.
A.~C.\ acknowledges funding from the Schmidt Futures Foundation.

\bibliography{bib}

\end{document}
